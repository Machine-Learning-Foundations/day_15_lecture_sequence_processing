\documentclass{standalone}
\usepackage{tikz}
\usetikzlibrary{positioning,arrows,calc}
\begin{document}
\pagestyle{empty}

\def\layersep{2.0cm}

\begin{tikzpicture}[shorten >=1pt,->,draw=black!50, node distance=\layersep,transform shape,rotate=90]  %<-- rotate the NN
  \tikzstyle{every pin edge}=[<-,shorten <=1pt]
  \tikzstyle{neuron}=[circle,fill=black!25,minimum size=17pt,inner sep=0pt]
	\tikzstyle{attent neuron}=[neuron, fill=green!50];
	\tikzstyle{rcNeuron}=[rectangle,fill=black!25,minimum size=17pt,inner sep=0pt]
	\tikzstyle{BLSTM rcNeuron}=[rcNeuron, fill=red!50];
	\tikzstyle{LSTM rcNeuron}=[rcNeuron, fill=blue!50];


    \tikzstyle{annot} = [text width=4em, text centered]
    \tikzset{hoz/.style={rotate=-90}}   %<--- for labels
    % Draw the input layer nodes
    \foreach \pos [count=\name] in {0,...,7,9} {
        \ifnum \pos=9  %create the x_T input.
          \node[BLSTM rcNeuron, pin=above:
            \rotatebox{-90}{\parbox[t][][r]{8mm}{\centering $\mathbf{x}_{T}$}}]
            (I-\name) at (0,-\pos) {};
        \else
          \node[BLSTM rcNeuron, pin=above:
            \rotatebox{-90}{\parbox[t][][r]{8mm}{\centering $\mathbf{x}_{\name}$}}]
            (I-\name) at (0,-\pos) {};
        \fi
      }
    % Draw the hidden layer nodes
    \foreach \pos [count=\name] in {0.5,2.5,4.5,6.5,8.5}
        \node[BLSTM rcNeuron] (H-\name) at (\layersep,-\pos cm) {};

    % Draw the output layer nodes
    \foreach \pos [count=\name] in {1.5,5.5,8} {
      \node[BLSTM rcNeuron] (O-\name) at (\layersep*2,-\pos cm) {};
      %\node[output rcNeuron, pin=right:
      %  \rotatebox{-90}{\parbox[t][][r]{8mm}{\centering $h_{\name}$}}]
      %  (O-\name) at (\layersep*2,-\pos cm) {};
    }


  	%draw the input layer recurrent connections.
  	\foreach \startVal [count=\endVal] in {2,...,8} {
       \draw[transform canvas={yshift=0.4ex},->] (I-\endVal) -- (I-\startVal);
       \draw[transform canvas={yshift=-0.4ex},->] (I-\startVal) -- (I-\endVal);
  	}
    \draw[dashed, transform canvas={yshift=0.4ex},->] (I-8) -- (I-9);
    \draw[dashed, transform canvas={yshift=-0.4ex},->] (I-9) -- (I-8);

    %draw the hidden layer recurrent connections.
    \foreach \startVal [count=\endVal] in {2,...,4} {
    \draw[transform canvas={yshift=0.4ex},->] (H-\endVal) -- (H-\startVal);
    \draw[transform canvas={yshift=-0.4ex},->] (H-\startVal) -- (H-\endVal);
  	}
    \draw[dashed, transform canvas={yshift=0.4ex},->] (H-4) -- (H-5);
    \draw[dashed, transform canvas={yshift=-0.4ex},->] (H-5) -- (H-4);

    %draw the output layer recurrent connections.
    \draw[transform canvas={yshift=0.4ex},->] (O-1) -- (O-2);
    \draw[transform canvas={yshift=-0.4ex},->] (O-2) -- (O-1);
    \draw[dashed, transform canvas={yshift=0.4ex},->] (O-2) -- (O-3);
    \draw[dashed, transform canvas={yshift=-0.4ex},->] (O-3) -- (O-2);

    %Connect connect input and hidden layer.
    %\foreach \source [evaluate=\source as \dest using {\source/2+0.5}]in {1,...,7} {
    %  \draw[->] (H-\dest) -- (I-\source);
    %}
    \draw[->] (I-1) -- (H-1);
    \draw[->] (I-2) -- (H-1);
    \draw[->] (I-3) -- (H-2);
    \draw[->] (I-4) -- (H-2);
    \draw[->] (I-5) -- (H-3);
    \draw[->] (I-6) -- (H-3);
    \draw[->] (I-7) -- (H-4);
    \draw[->] (I-8) -- (H-4);
    \draw[->] (I-9) -- (H-5);

    % Connect the hidden layer to the output layer
    \draw[->] (H-1) -- (O-1);
    \draw[->] (H-2) -- (O-1);
    \draw[->] (H-3) -- (O-2);
    \draw[->] (H-4) -- (O-2);
    \draw[->] (H-5) -- (O-3);

    % Annotate the layers
    \node[annot,above of=H-1, node distance=1cm,hoz] (hl) {}; %Hidden layer
    %\node[annot,left of=hl,hoz] {Input layer};
    \node[annot,right of=hl,hoz] {Listener:};

    % draw the listener output nodes
    \node[right of=O-1, node distance=1.0cm, rotate=-90] (h1) {$\mathbf{h}_1$};
    \draw[->] (O-1) -- (h1);
    \node[right of=O-2, node distance=1.0cm, rotate=-90] (h2) {$\mathbf{h}_2$};
    \draw[->] (O-2) -- (h2);
    \node[right of=O-3, node distance=1.0cm, rotate=-90] (h3) {$\mathbf{h}_u$};
    \draw[->] (O-3) -- (h3);

    % draw the node for the h vector.
    \node[below right of=h1, rotate=-90, node distance=1.5cm] (h)
        {$\mathbf{H} = (\mathbf{h}_1, \mathbf{h}_2, \dots ,\mathbf{h}_u)$};
    \draw[-] (h1) -- (h);
    \draw[-] (h2) -- (h);
    \draw[-] (h3) -- (h);

    %draw the speller bottom layer.
    \foreach \pos [count=\name] in {0,...,2,4} {
        \ifnum \pos=4.5  %create the x_T input.
          \node[LSTM rcNeuron, pin=above:
            \rotatebox{-90}{\parbox[t][][r]{3mm}{\centering $\mathbf{y}_{s-1}$}}]
            (SI-\name) at (\layersep*4,-\pos*2) {};
        \else
          \ifnum \pos=0
            \node[LSTM rcNeuron, pin=above:
              \rotatebox{-90}{\parbox[t][][r]{3mm}{\centering $<$sos$>$}}]
              (SI-\name) at (\layersep*4,-\pos*2) {};

          \else
            \node[LSTM rcNeuron, pin=above:
              \rotatebox{-90}{\parbox[t][][r]{3mm}{\centering $\mathbf{y}_{\name}$}}]
              (SI-\name) at (\layersep*4,-\pos*2) {};
          \fi
        \fi
      }

      %draw the attent layer.
      \foreach \pos [count=\name] in {0,1,2,4} {
        \node[attent neuron]
          (A-\name) at (\layersep*5,-\pos*2-0.5) {};
      }

      %draw the third speller layer.
      \foreach \pos [count=\name] in {0,...,2,4} {
        \node[LSTM rcNeuron]
          (SH-\name) at (\layersep*6,-\pos*2-0.5) {};
      }

      %draw the speller output layer.
      \foreach \pos [count=\name] in {0,...,2,4} {
        \node[LSTM rcNeuron]
          (SO-\name) at (\layersep*7,-\pos*2-0.5) {};
      }

      %draw the output distributions.
      \node[right of=SO-1, node distance=1.0cm, rotate=-90] (y2) {$\mathbf{y}_2$};
      \draw[->] (SO-1) -- (y2);
      \node[right of=SO-2, node distance=1.0cm, rotate=-90] (y3) {$\mathbf{y}_3$};
      \draw[->] (SO-2) -- (y3);
      \node[right of=SO-3, node distance=1.0cm, rotate=-90] (y4) {$\mathbf{y}_4$};
      \draw[->] (SO-3) -- (y4);
      \node[right of=SO-4, node distance=1.0cm, rotate=-90] (eos) {$<$eos$>$};
      \draw[->] (SO-4) -- (eos);

      %draw the sideward speller connections.
      \foreach \endVal [count=\startVal] in {2,...,3} {
         \draw[->] (SH-\startVal) -- (SH-\endVal);
         \draw[->] (SI-\startVal) -- (SI-\endVal);
      }
      \draw[dashed,->] (SI-3) -- (SI-4);
      \draw[dashed,->] (SH-3) -- (SH-4);

      %draw the forward hidden speller connections.
      \foreach \val in {1,...,4} {
         \draw[->] (SH-\val) -- (SO-\val);
      }

      %attend forward
      \draw[->] (SI-1) -- (A-1);
      \draw[->] (SI-2) -- (A-2);
      \draw[->] (SI-3) -- (A-3);
      \draw[->] (SI-4) -- (A-4);
      \draw[->] (A-1) -- (SH-1);
      \draw[->] (A-2) -- (SH-2);
      \draw[->] (A-3) -- (SH-3);
      \draw[->] (A-4) -- (SH-4);

      %attend connections.
      \draw[->, bend angle=10, bend left] (h) to (A-1) node[below right, rotate=-90] {$\mathbf{H}$};
      \draw[->, bend angle=20, bend right] (h) to (A-2) node[below right, rotate=-90] {$\mathbf{H}$};
      \draw[->, bend angle=25, bend right] (h) to (A-4) node[below left, rotate=-90] {$\mathbf{H}$};
      \draw[->, bend angle=55, bend right] (h) to (A-3) node[below right, 	
      rotate=-90] {$\mathbf{H}$};

      %draw the c vector connections.
      \draw[->, bend angle=45, bend left] (A-1) to ($(SI-2) + (0.3,0.25)$) node[above left, rotate=-90] {$\mathbf{c}_1$};
      \draw[->, bend angle=45, bend left] (A-2) to ($(SI-3) + (0.3,0.25)$) node[above left, rotate=-90] {$\mathbf{c}_2$};
      \draw[dashed, ->, bend angle=45, bend left] (A-3) to ($(SI-4) + (0.3,0.25)$) node[above left, rotate=-90] {};

      \node[above right of=y2, node distance=1 cm, rotate=-90] {Speller:};
      %\node[annot, below of=A-3, node distance=1.4 cm, hoz] {Attention-Context creates \\
      %                                                        context vector $c_i$ from $h$ \\
      %                                                        and $s_i$};
      %\node[annot, below of=SO-4, node distance=1.4 cm, hoz] {Grapheme characters $y_i$ \\
      %                                                       are modelled by the
      %                                                       CharacterDistribution};
;

\end{tikzpicture}
% End of code
\end{document}